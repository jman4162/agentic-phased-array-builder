\documentclass{IEEEtran}
\usepackage{cite}
\usepackage{amsmath,amssymb,amsfonts}
\usepackage{graphicx}
\usepackage{textcomp}
\usepackage{booktabs}
\usepackage{siunitx}
\usepackage{bm}
\usepackage{url}
\usepackage{draftwatermark}
\SetWatermarkText{DRAFT}
\SetWatermarkScale{1.5}
\SetWatermarkColor[gray]{0.85}

\sisetup{per-mode=symbol}

\graphicspath{{output/}}

\def\BibTeX{{\rm B\kern-.05em{\sc i\kern-.025em b}\kern-.08em
    T\kern-.1667em\lower.7ex\hbox{E}\kern-.125emX}}

\begin{document}

\title{Integrated Design and System-Level Analysis of a 28\,GHz
       Phased-Array Antenna for 5G mmWave Communications}

\author{John~A.~Hodge
\thanks{Manuscript prepared February 2026. This work was supported by the
Agentic Phased Array Builder (APAB) open-source project.}
\thanks{J.~A.~Hodge is with Virginia Polytechnic Institute and State
University, Blacksburg, VA 24061 USA
(e-mail: jah70@vt.edu).}}

\maketitle

\begin{abstract}
This paper presents a complete, integrated design methodology for a
28\,GHz millimeter-wave phased-array antenna intended for 5G New Radio
(NR) fixed wireless access. The methodology spans five analysis domains
within a single automated pipeline: unit-cell impedance characterization
via full-wave finite-element simulation (EdgeFEM), far-field pattern synthesis with Taylor
amplitude tapering, mutual-coupling prediction through a
distance-dependent scattering-matrix model, active-impedance and
scan-blindness assessment, and system-level link-budget evaluation for a
representative urban deployment scenario. A design-of-experiments trade
study using Latin Hypercube Sampling explores a three-dimensional
parameter space---array dimensions $N_x$, $N_y$ and per-element
transmit power---yielding 40 candidate architectures whose cost, EIRP,
and received SNR are compared. The baseline $8\times8$ configuration on
Rogers RO4003C ($\varepsilon_r = 3.55$, $h = 0.254$\,mm)
achieves a $-10$\,dB impedance bandwidth of 800\,MHz
centered at 28.4\,GHz, a broadside directivity of
10.48\,dBi, an EIRP of 30.1\,dBW, and a link margin of
27.6\,dB at 200\,m range with
400\,MHz channel bandwidth. The mean active reflection
coefficient across all 64~elements is $\lvert\Gamma\rvert = 0.265$,
corresponding to a mismatch loss of only 0.40\,dB. All
analysis steps and figures are generated by the open-source APAB
toolkit, demonstrating a reproducible, LLM-orchestrated workflow from
unit-cell physics to system-level feasibility.
\end{abstract}

\begin{IEEEkeywords}
5G NR, active impedance, design of experiments, link budget,
millimeter-wave antennas, mutual coupling, phased arrays, trade study
\end{IEEEkeywords}

\section{Introduction}
\label{sec:intro}

\IEEEPARstart{T}{he} commercial deployment of 5G New Radio in the millimeter-wave
spectrum---particularly the $n257$ and $n261$ bands centered near
\SI{28}{\giga\hertz}---has renewed interest in planar phased-array
antennas that can provide the electronic beam-steering gain required to
overcome the severe free-space path loss at these frequencies
\cite{balanis2016,mailloux2017}. A single-element microstrip patch at
\SI{28}{\giga\hertz} offers roughly \SIrange{5}{7}{dBi} of gain, far
short of the \SIrange{20}{25}{\deci\bel} of effective isotropic
radiated power (EIRP) demanded by the 3GPP power-class specifications
for base-station and fixed-wireless-access equipment. Arrays of 64 to
256~elements therefore form the standard building block, and the design
of such arrays requires a careful chain of analysis that links
electromagnetic unit-cell behavior to system-level communication
performance.

In practice, each stage of this analysis is often performed with
separate tools, distinct data formats, and ad-hoc scripts that are
difficult to reproduce or to vary parametrically. The present work
demonstrates an alternative approach in which all stages---unit-cell
impedance modeling, array-factor synthesis, mutual-coupling estimation,
active-impedance computation, link-budget evaluation, and
multi-objective trade-study optimization---are executed within a single
automated pipeline implemented by the Agentic Phased Array Builder
(APAB) toolkit \cite{apab2026}. APAB exposes each analysis capability
as a Model Context Protocol (MCP) tool that can be invoked
programmatically or orchestrated by a large language model, enabling
rapid exploration of the design space with full provenance tracking.

The remainder of the paper is organized as follows.
Section~\ref{sec:unitcell} presents the full-wave unit-cell simulation and
validates its impedance and bandwidth predictions.
Section~\ref{sec:pattern} describes the array-factor formulation and
the Taylor amplitude taper used to control sidelobes.
Section~\ref{sec:coupling} constructs the mutual-coupling scattering
matrix and derives the active element impedance and reflection
coefficients. Section~\ref{sec:system} evaluates the complete
link budget for a 5G NR fixed-wireless scenario.
Section~\ref{sec:trade} presents the design-of-experiments trade study
and its Pareto-optimal results. Section~\ref{sec:conclusion} summarizes
the key findings.

% ======================================================================
\section{Unit-Cell Impedance Characterization}
\label{sec:unitcell}

\subsection{Geometry and Substrate}

The radiating element is a rectangular microstrip patch of width
$W = \SI{3.8}{\milli\meter}$ and length $L = \SI{2.7}{\milli\meter}$,
printed on a \SI{0.254}{\milli\meter}-thick Rogers RO4003C substrate
with relative permittivity $\varepsilon_r = 3.55$ and loss tangent
$\tan\delta = 0.0027$. The unit-cell period in both the $x$- and
$y$-directions is set to one half free-space wavelength at the design
frequency,
\begin{equation}
  d_x = d_y = \frac{\lambda_0}{2}
             = \frac{c}{2 f_0}
             = \SI{5.36}{\milli\meter},
  \label{eq:spacing}
\end{equation}
which places the grating-lobe onset at $\theta = \ang{90}$ for
broadside operation and avoids scan blindness within the intended
$\pm\ang{60}$ field of regard.

\subsection{Full-Wave FEM Simulation}

The unit-cell electromagnetic response is computed using EdgeFEM
\cite{edgefem2026}, a 3-D finite-element solver that employs
N\'{e}d\'{e}lec (edge) basis functions on tetrahedral meshes. The solver
enforces Floquet periodic boundary conditions on the $\pm x$ and $\pm y$
faces of the unit cell, with Floquet wave ports at the top and bottom
boundaries to excite and absorb the fundamental plane-wave mode. This
formulation naturally captures higher-order mode interactions,
substrate-surface-wave coupling, and patch-edge fringing effects that
are absent from analytical cavity models.

The mesh is generated at a density of 3~elements per wavelength at the
design frequency to keep runtime manageable (approximately 5~minutes for
the full sweep on a single CPU core). A frequency sweep from
\SIrange{26}{30}{\giga\hertz} with 21~points produces the complex
reflection coefficient $\Gamma(f)$ at each frequency. For structures
with a PEC ground plane, the transmission coefficient is identically
zero and all incident power is either reflected or dissipated.

Because the unit cell includes a PEC ground plane, the Floquet
reflection magnitude $|\Gamma|$ remains near unity across the band;
the patch resonance manifests as a rapid transition in the reflection
\emph{phase}. The maximum phase derivative occurs near
\SI{28.5}{\giga\hertz}, confirming the patch resonance location via
full-wave simulation. Scan-angle dependence is evaluated by solving
additional single-frequency problems at \SI{28}{\giga\hertz} with the
Floquet phase shift set for incidence angles of \ang{0}, \ang{15},
\ang{30}, \ang{45}, and \ang{60}. The surface impedance
$Z_s = Z_0 (1 + \Gamma) / (1 - \Gamma) = j\SI{-9.1}{\ohm}$ is
extracted at the design frequency for use in subsequent coupling
analysis.

\subsection{Analytical Validation: Effective Permittivity and Resonant Frequency}

The effective permittivity of the microstrip line is computed from the
classical quasi-static expression \cite{hammerstad1975},
\begin{equation}
  \varepsilon_\mathrm{eff}
    = \frac{\varepsilon_r + 1}{2}
    + \frac{\varepsilon_r - 1}{2}
      \left(1 + 12\,\frac{h}{W}\right)^{-1/2},
  \label{eq:epseff}
\end{equation}
which yields $\varepsilon_\mathrm{eff} = 3.225$ for the present
geometry. The physical patch length is extended by the Hammerstad
fringing correction \cite{hammerstad1975},
\begin{equation}
  \Delta L = 0.412\,h\;
    \frac{(\varepsilon_\mathrm{eff} + 0.3)(W/h + 0.264)}
         {(\varepsilon_\mathrm{eff} - 0.258)(W/h + 0.8)},
  \label{eq:fringing}
\end{equation}
giving $\Delta L = \SI{0.120}{\milli\meter}$. The resonant frequency of
the dominant $\mathrm{TM}_{010}$ mode is then
\begin{equation}
  f_0 = \frac{c}{2(L + 2\Delta L)\sqrt{\varepsilon_\mathrm{eff}}}
      = \SI{28.4}{\giga\hertz}.
  \label{eq:fres}
\end{equation}

\subsection{Analytical Validation: Edge Impedance and Inset-Feed Matching}

The radiation conductance at the patch edge is approximated by the
Bahl--Bhartia formula \cite{bahl1980},
\begin{equation}
  G_\mathrm{rad}
    = \frac{W}{120\,\lambda_0}
      \left[1 - \frac{(k_0 h)^2}{24}\right],
  \label{eq:grad}
\end{equation}
where $k_0 = 2\pi f_0 / c$ is the free-space wavenumber. The edge
resistance follows as $R_\mathrm{edge} = 1/(2\,G_\mathrm{rad}) =
\SI{169.3}{\ohm}$. A coaxial-probe or microstrip inset feed recessed
by a distance $y_0$ from the radiating edge transforms the input
impedance according to
\begin{equation}
  Z_\mathrm{in}(y_0)
    = R_\mathrm{edge}\,\cos^2\!\left(\frac{\pi\,y_0}{L}\right).
  \label{eq:inset}
\end{equation}
Setting $Z_\mathrm{in} = Z_0 = \SI{50}{\ohm}$ and solving for the
inset depth yields
\begin{equation}
  y_0 = \frac{L}{\pi}
        \arccos\!\sqrt{\frac{Z_0}{R_\mathrm{edge}}}
      = \SI{0.856}{\milli\meter}
  \label{eq:y0}
\end{equation}
or approximately 31.7\% of the patch length.

\subsection{Analytical Validation: Quality Factor and Impedance Bandwidth}

The loaded quality factor governs the impedance bandwidth. Radiation and
dielectric losses contribute independently:
\begin{equation}
  Q_\mathrm{rad}
    = \frac{c}{4\,f_0\,h}\,\sqrt{\varepsilon_\mathrm{eff}},
  \qquad
  Q_d = \frac{1}{\tan\delta},
  \label{eq:qfactors}
\end{equation}
yielding $Q_\mathrm{rad} = 18.7$ and $Q_d = 370$. The total quality
factor is
\begin{equation}
  \frac{1}{Q_T}
    = \frac{1}{Q_\mathrm{rad}} + \frac{1}{Q_d},
  \qquad
  Q_T = 17.8.
  \label{eq:qtotal}
\end{equation}
For the thin, low-loss substrate considered here, radiation loss
dominates and the dielectric contribution is negligible.

\subsection{Analytical Validation: Frequency-Dependent Input Impedance}

The input impedance of the matched patch is modeled as a parallel RLC
resonator observed at the inset-feed point,
\begin{equation}
  Z_\mathrm{in}(f)
    = \frac{R_\mathrm{feed}}
           {1 + j\,2\,Q_T\,\dfrac{f - f_0}{f_0}}
      \;\bigl(1 - j\,\tan\delta\bigr),
  \label{eq:zin}
\end{equation}
where $R_\mathrm{feed} \approx \SI{50}{\ohm}$ by construction and the
multiplicative loss factor accounts for substrate dissipation. The
corresponding reflection coefficient referred to $Z_0 = \SI{50}{\ohm}$
is
\begin{equation}
  \Gamma(f) = \frac{Z_\mathrm{in}(f) - Z_0}{Z_\mathrm{in}(f) + Z_0},
  \label{eq:s11}
\end{equation}
and the $-10$\,dB return-loss bandwidth is defined as the contiguous
frequency range over which $20\log_{10}|\Gamma| < -\SI{10}{\deci\bel}$.

Figure~\ref{fig:unitcell} presents the combined characterization results.
The left panel shows the analytical feed-port return loss over the
\SIrange{26}{30}{\giga\hertz} sweep band, predicting a resonance at
\SI{28.4}{\giga\hertz} with a minimum return loss of
$-\SI{46.7}{\deci\bel}$ and a $-10$\,dB bandwidth of
\SI{800}{\mega\hertz} (\SIrange{28.0}{28.8}{\giga\hertz}). The center
panel shows the EdgeFEM Floquet reflection phase versus frequency; the
rapid phase transition near \SI{28.5}{\giga\hertz} independently
confirms the resonance location predicted by the analytical model,
with less than 0.4\% frequency discrepancy. The right panel reports
the analytical scan-angle variation of feed-port $|S_{11}|$ at
resonance; the return loss degrades from $-\SI{57}{\deci\bel}$ at
broadside to $-\SI{9.5}{\deci\bel}$ at $\theta = \ang{60}$, consistent
with the $1/\cos\theta$ impedance scaling of Floquet-mode analysis.

\begin{figure*}[!t]
  \centering
  \includegraphics[width=\textwidth]{fig1_fem_sparams.png}
  \caption{Unit-cell characterization of the \SI{28}{\giga\hertz}
           microstrip patch (EdgeFEM + analytical).
           \textbf{Left:} Feed-port return loss versus frequency from
           the analytical cavity model, showing the resonance at
           \SI{28.4}{\giga\hertz} and \SI{800}{\mega\hertz} impedance
           bandwidth.
           \textbf{Center:} EdgeFEM Floquet reflection phase versus
           frequency, showing the phase transition at the patch
           resonance that independently confirms the analytical
           prediction.
           \textbf{Right:} Feed-port return loss at resonance as a
           function of scan angle $\theta$, illustrating the impedance
           mismatch growth at wide scan.}
  \label{fig:unitcell}
\end{figure*}

% ======================================================================
\section{Array Pattern Synthesis}
\label{sec:pattern}

\subsection{Array Factor Formulation}

The far-field pattern of a planar array of $N = N_x \times N_y$
isotropic elements arranged on a rectangular lattice with spacings
$d_x$ and $d_y$ is given by the array factor
\begin{equation}
  \mathrm{AF}(\theta, \phi)
    = \sum_{m=0}^{N_x - 1}\;\sum_{n=0}^{N_y - 1}
      w_{mn}\,
      e^{\,j\,k_0\bigl[m\,d_x(\sin\theta\cos\phi - \sin\theta_0\cos\phi_0)
                      + n\,d_y(\sin\theta\sin\phi - \sin\theta_0\sin\phi_0)\bigr]},
  \label{eq:af}
\end{equation}
where $(\theta_0, \phi_0)$ is the desired scan direction and $w_{mn}$
is the complex excitation weight for the $(m,n)$-th element. The total
radiated pattern is the product of the array factor and the embedded
element pattern; in the present analysis the element pattern is taken as
isotropic, which is a reasonable first approximation for the broadside
and moderate-scan cases considered.

\subsection{Taylor Amplitude Taper}

To suppress sidelobes below $-\SI{25}{\deci\bel}$, a Taylor $\bar{n}$
window \cite{taylor1955} is applied to the element weights along each
principal axis. The Taylor window is separable in the $x$- and
$y$-dimensions,
\begin{equation}
  w_{mn} = w_m^{(x)}\;w_n^{(y)}\;e^{j\,\psi_{mn}},
  \label{eq:taper}
\end{equation}
where $\psi_{mn}$ encodes the progressive phase shift for beam
steering. The taper trades a modest reduction in directivity---and a
corresponding broadening of the main beam---against substantially
improved sidelobe suppression relative to the uniform-excitation case.

For the baseline $8\times8$ array at broadside, the Taylor-tapered
pattern achieves a directivity of \SI{10.48}{dBi} with a half-power
beamwidth of \ang{16} in the E-plane. When the beam is steered to
$\theta_0 = \ang{15}$ in the E-plane, the directivity decreases by only
\SI{0.02}{\deci\bel}, indicating that scan loss is negligible at this
modest deflection angle.

Figure~\ref{fig:pattern} presents four complementary views of the
array radiation pattern. The E-plane and H-plane polar plots show the
principal-plane pattern shapes with their characteristic sidelobe
structure. The UV-space representation maps the pattern onto the
direction-cosine plane, where the visible-region circle at
$\sqrt{u^2 + v^2} = 1$ clearly delineates real-space radiation from
evanescent modes, and grating-lobe circles confirm that half-wavelength
spacing keeps all grating lobes outside the visible region. The 3-D
surface rendering provides an intuitive volumetric view of the beam
shape. Interactive Plotly versions of the 3-D pattern, UV-space map,
and an animated beam-scanning visualization are also generated by the
pipeline for detailed exploration.

\begin{figure*}[!t]
  \centering
  \includegraphics[width=\textwidth]{fig2_array_pattern.png}
  \caption{Array radiation pattern for the $8\times8$ Taylor-tapered
           array at \SI{28}{\giga\hertz}.
           \textbf{(a)}~E-plane polar pattern at broadside (solid) and
           steered to $\theta_0 = \ang{15}$ (dashed).
           \textbf{(b)}~H-plane polar pattern at broadside.
           \textbf{(c)}~UV-space pattern with visible-region circle and
           grating-lobe positions for $d = \lambda/2$ spacing.
           \textbf{(d)}~3-D surface rendering of the radiation pattern
           on the unit sphere, colored by normalized gain in decibels.}
  \label{fig:pattern}
\end{figure*}

% ======================================================================
\section{Mutual Coupling and Active Impedance}
\label{sec:coupling}

\subsection{Scattering-Matrix Construction}

In a finite phased array, electromagnetic coupling between neighboring
elements modifies the port impedance seen by each transmit/receive
module and perturbs the far-field pattern. The coupling is described by
the $N \times N$ scattering matrix $\mathbf{S}$, where the diagonal
entries $S_{ii}$ represent the isolated-element reflection coefficient
and the off-diagonal entries $S_{ij}$ ($i \neq j$) capture the mutual
coupling between ports $i$ and $j$.

The diagonal is populated with the analytically computed $S_{11}$ at
resonance (\SI{28.4}{\giga\hertz}), which equals $-\SI{46.7}{\deci\bel}$
in magnitude---consistent with the EdgeFEM Floquet analysis that
independently confirms the resonance location. For the off-diagonal
terms, a distance-dependent coupling
model calibrated to published measurements of millimeter-wave patch
arrays \cite{pozar1994} is adopted:
\begin{equation}
  |S_{ij}|_{\mathrm{dB}}
    = -18 - 6\,(d_{ij} - 1),
  \qquad
  d_{ij} \leq 2.1,
  \label{eq:coupling}
\end{equation}
where $d_{ij}$ is the Euclidean distance between elements $i$ and $j$
measured in units of the lattice constant. Direct neighbors
($d_{ij} = 1$) couple at $-\SI{18}{\deci\bel}$, diagonal neighbors
($d_{ij} = \sqrt{2} \approx 1.41$) at approximately
$-\SI{20.5}{\deci\bel}$, and second neighbors ($d_{ij} = 2$) at
$-\SI{24}{\deci\bel}$. Elements separated by more than 2.1 lattice
constants are treated as uncoupled. The phase of each coupling
coefficient is drawn from a uniform distribution on $[-\pi, \pi]$ to
represent the quasi-random phase variation observed in practice.
The resulting $64 \times 64$ matrix contains 612 nonzero off-diagonal
entries with a mean coupling magnitude of $-\SI{20.3}{\deci\bel}$.

\subsection{Active Reflection Coefficient}

When all elements are excited simultaneously with excitation vector
$\mathbf{a} \in \mathbb{C}^N$, the active reflection coefficient at
port~$i$ is
\begin{equation}
  \Gamma_i^\mathrm{act}
    = \sum_{j=1}^{N} S_{ij}\,\frac{a_j}{a_i},
  \label{eq:gamma_active}
\end{equation}
or in matrix notation, $\bm{\Gamma}^\mathrm{act} =
\mathrm{diag}(\mathbf{a})^{-1}\,\mathbf{S}\,\mathbf{a}$. For uniform
excitation ($a_j = 1$ for all $j$), this simplifies to the row sums of
the scattering matrix. The active impedance is recovered through
\begin{equation}
  Z_i^\mathrm{act}
    = Z_0\,\frac{1 + \Gamma_i^\mathrm{act}}
                 {1 - \Gamma_i^\mathrm{act}}.
  \label{eq:zact}
\end{equation}

The per-element mismatch loss, representing the fraction of incident
power reflected back toward the transmit module, is
\begin{equation}
  L_\mathrm{mm}
    = -10\log_{10}\!\Bigl(1 - \overline{|\Gamma^\mathrm{act}|^2}\Bigr),
  \label{eq:mismatch}
\end{equation}
where the overline denotes averaging across all $N$ elements.

Scan blindness is detected by searching for elements whose active
reflection coefficient magnitude approaches or exceeds unity
($|\Gamma_i^\mathrm{act}| \geq 0.95$), which would indicate a
surface-wave resonance that traps power within the array aperture
\cite{pozar1984}.

\subsection{Results}

The computed active reflection coefficient magnitudes span the range
$[0.045, 0.718]$ with a mean of $\overline{|\Gamma|} = 0.265$, yielding
a mismatch loss of \SI{0.395}{\deci\bel} (computed from the mean-square
reflection $\overline{|\Gamma|^2}$ per \eqref{eq:mismatch}). No scan-blindness points are
detected. The active impedance real parts range from \SIrange{8.2}{102.2}{\ohm},
centered near the design impedance of \SI{50}{\ohm}, confirming that the
array environment does not catastrophically detune the individual
elements.

Figure~\ref{fig:coupling} presents the coupling analysis in three
panels. The left panel displays the full $64\times64$ scattering matrix
in decibels, clearly showing the banded structure arising from the
distance-dependent coupling law. The center panel maps the active
$|\Gamma|$ for each element onto the $8\times8$ grid, revealing that
edge and corner elements experience the lowest coupling (fewest
neighbors) while interior elements, on average, accumulate more
reflected energy due to the larger number of coupling paths. The
right panel plots the active impedance of each element on the complex
$Z$-plane, colored by $|\Gamma|$; most points cluster near
\SI{50}{\ohm} with moderate reactive components, and the outliers
correspond to the elements with the highest mutual-coupling
contributions.

\begin{figure*}[!t]
  \centering
  \includegraphics[width=\textwidth]{fig3_coupling_analysis.png}
  \caption{Mutual coupling analysis for the $8\times8$ array.
           \textbf{Left:} Scattering-parameter matrix magnitude (dB).
           \textbf{Center:} Active reflection coefficient magnitude per
           element, mapped onto the array grid.
           \textbf{Right:} Active impedance on the complex plane,
           colored by $|\Gamma^\mathrm{act}|$.}
  \label{fig:coupling}
\end{figure*}

\subsection{Effect of Coupling on the Radiation Pattern}

The coupling-perturbed excitation vector modifies the effective weights
and consequently the array factor. When the scattering matrix is
incorporated into the pattern computation through the relation
$\mathbf{a}_\mathrm{eff} = (\mathbf{I} + \mathbf{S})\,\mathbf{a}$,
the coupled directivity is \SI{16.13}{dBi}, representing an apparent
increase of \SI{5.65}{\deci\bel} over the isolated-element value.
This increase is an artifact of the uniform-excitation assumption
combined with the constructive interference of the coupling terms in
the broadside direction; in practice, the additional radiated power
originates from mutual coupling rather than from the intended
excitation, and the realized gain at the system level remains governed
by the EIRP computed from the power amplifier output and the array
gain.

% ======================================================================
\section{System-Level Link Budget}
\label{sec:system}

The array and unit-cell parameters are propagated into a complete 5G~NR
downlink link budget for a fixed-wireless-access scenario. The system
parameters are summarized in Table~\ref{tab:linkparams}.

\begin{table}[!t]
  \centering
  \caption{Link-Budget Parameters}
  \label{tab:linkparams}
  \begin{tabular}{lS[table-format=3.2]l}
    \toprule
    Parameter & {Value} & Unit \\
    \midrule
    Carrier frequency       & 28.0   & GHz \\
    Channel bandwidth       & 400    & MHz \\
    Array size ($N_x \times N_y$)  & {$8 \times 8$} & \\
    Total elements          & 64     & \\
    TX power per element    & 0.10   & W \\
    Element spacing         & 5.36   & mm \\
    Range                   & 200    & m \\
    Required SNR            & 10     & dB \\
    \bottomrule
  \end{tabular}
\end{table}

The total transmit power is $P_\mathrm{tx} = N \cdot P_\mathrm{elem} =
64 \times \SI{0.1}{\watt} = \SI{6.4}{\watt}$ (\SI{8.06}{dBW}). The
array gain---incorporating both the element pattern and the
Taylor-tapered array factor---is
\SI{23.03}{\deci\bel}, giving an EIRP of
\begin{equation}
  \mathrm{EIRP}
    = P_\mathrm{tx,dBW} + G_\mathrm{array,dB}
    = 8.06 + 23.03
    = \SI{30.10}{dBW}.
  \label{eq:eirp}
\end{equation}

Free-space path loss at \SI{28}{\giga\hertz} over \SI{200}{\meter} is
\begin{equation}
  \mathrm{FSPL}
    = 20\log_{10}\!\left(\frac{4\pi R f}{c}\right)
    = \SI{107.4}{\deci\bel}.
  \label{eq:fspl}
\end{equation}

Assuming an isotropic receive antenna ($G_\mathrm{rx} = \SI{0}{dBi}$),
the received power is
\begin{equation}
  P_\mathrm{rx}
    = \mathrm{EIRP} - \mathrm{FSPL} + G_\mathrm{rx}
    = 30.10 - 107.41
    = \SI{-77.3}{dBW}.
  \label{eq:prx}
\end{equation}

The thermal noise power in a \SI{400}{\mega\hertz} bandwidth with a
system noise figure of \SI{5}{\deci\bel} is
\begin{equation}
  P_N
    = k_B T_0 B \cdot F
    = \SI{-115.0}{dBW},
  \label{eq:noise}
\end{equation}
where $k_B = \SI{1.38e-23}{\joule\per\kelvin}$,
$T_0 = \SI{290}{\kelvin}$, $B = \SI{400}{\mega\hertz}$, and
$F = \SI{5}{\deci\bel}$. The received signal-to-noise ratio is therefore
\begin{equation}
  \mathrm{SNR}_\mathrm{rx}
    = P_\mathrm{rx} - P_N
    = -77.3 - (-115.0)
    = \SI{37.6}{\deci\bel},
  \label{eq:snr}
\end{equation}
yielding a link margin of $37.6 - 10.0 = \SI{27.6}{\deci\bel}$ above the
required \SI{10}{\deci\bel} SNR threshold. This substantial margin
allows for rain attenuation, atmospheric absorption, polarization
mismatch, implementation losses, and additional fading that are not
included in this first-order analysis. The estimated hardware cost for
the 64-element array is \$6,400 at \$100 per element, and the total DC
power consumption is \SI{21.3}{\watt}.

Figure~\ref{fig:linkbudget} presents a waterfall chart of the
link-budget components, providing an intuitive visualization of how each
factor contributes to the final SNR.

\begin{figure}[!t]
  \centering
  \includegraphics[width=\columnwidth]{fig4_link_budget.png}
  \caption{Link-budget waterfall for the $8\times8$ array at
           \SI{200}{\meter} range. Positive contributions (blue)
           include transmit power, array gain, and the resulting EIRP
           and SNR. Negative contributions (red) include path loss,
           taper loss, and noise floor.}
  \label{fig:linkbudget}
\end{figure}

% ======================================================================
\section{Design-of-Experiments Trade Study}
\label{sec:trade}

A systematic exploration of the design space is carried out using a
Latin Hypercube Sampling (LHS) design of experiments with 40 sample
points drawn from a three-dimensional parameter space. LHS ensures
that the projections of the sample points onto each parameter axis are
uniformly distributed \cite{mckay1979}, providing better coverage of
the design space than simple random sampling for the same number of
evaluations.

\subsection{Design Variables}

The three design variables and their ranges are: the number of
elements in the $x$-direction, $N_x \in [4, 16]$ (integer); the
number of elements in the $y$-direction, $N_y \in [4, 16]$ (integer);
and the per-element transmit power, $P_\mathrm{elem} \in
[\SI{10}{\milli\watt}, \SI{500}{\milli\watt}]$ (continuous). All other
parameters---frequency, substrate, patch dimensions, range, bandwidth,
and required SNR---are held at their baseline values. Each candidate
architecture is evaluated using the same link-budget model described in
Section~\ref{sec:system}.

\subsection{Metrics}

Three system-level metrics are computed for each design point: the EIRP
in dBW, which measures the radiated power in the beam direction and
scales with both array size and per-element power; the received SNR in
dB at \SI{200}{\meter} range, which determines whether the link closes;
and the estimated hardware cost in USD, which scales linearly with the
total number of elements.

\subsection{Results}

All 40 LHS sample points produce valid architectures, yielding a 100\%
evaluation success rate. The EIRP spans \SIrange{22.3}{43.7}{dBW}, the
received SNR ranges from \SI{29.8}{\deci\bel} to \SI{51.2}{\deci\bel},
and the cost varies from \$1,600 (for a $4\times4$ array) to \$25,600
(for a $16\times16$ array). Every design exceeds the \SI{10}{\deci\bel}
SNR requirement by a comfortable margin, indicating that even the
smallest arrays in the trade space provide adequate link performance for
the \SI{200}{\meter} scenario.

The top five Pareto-optimal designs ranked by EIRP are listed in
Table~\ref{tab:pareto}. The highest-EIRP design employs a
$13\times14$ array with \SI{280}{\milli\watt} per element, achieving
\SI{43.7}{dBW} at a cost of \$18,200. Notably, a $9\times15$ array
with \SI{430}{\milli\watt} per element achieves nearly the same EIRP
(\SI{42.9}{dBW}) at 26\% lower cost (\$13,500), illustrating the
value of systematic trade-study exploration in identifying
cost-effective design alternatives.

\begin{table}[!t]
  \centering
  \caption{Top Pareto-Optimal Designs by EIRP}
  \label{tab:pareto}
  \begin{tabular}{ccS[table-format=1.2]S[table-format=2.2]S[table-format=2.2]S[table-format=5.0]}
    \toprule
    $N_x$ & $N_y$ & {$P_\mathrm{elem}$ (W)} & {EIRP (dBW)} & {SNR (dB)} & {Cost (\$)} \\
    \midrule
    13 & 14 & 0.28 & 43.69 & 51.23 & 18200 \\
     9 & 15 & 0.43 & 42.91 & 50.46 & 13500 \\
    15 & 14 & 0.17 & 42.68 & 50.23 & 21000 \\
    14 & 10 & 0.30 & 41.70 & 49.25 & 14000 \\
     8 & 16 & 0.33 & 41.35 & 48.90 & 12800 \\
    \bottomrule
  \end{tabular}
\end{table}

Figure~\ref{fig:trade} presents the trade-study results in three
complementary views. The left panel shows EIRP versus total element
count with per-element power encoded as color, revealing the expected
monotonic increase in EIRP with both array size and transmit power. The
center panel plots EIRP against hardware cost, making the cost--performance
trade-off explicit: designs in the upper-left region of this plot
represent the most cost-efficient solutions. The right panel shows the
histogram of received SNR across all 40 designs, confirming that the
entire design space provides at least \SI{20}{\deci\bel} of margin above
the \SI{10}{\deci\bel} requirement.

\begin{figure*}[!t]
  \centering
  \includegraphics[width=\textwidth]{fig5_trade_study.png}
  \caption{Trade-study results for the three-variable design space.
           \textbf{Left:} EIRP versus total element count, colored by
           per-element transmit power.
           \textbf{Center:} EIRP versus hardware cost, colored by
           element count.
           \textbf{Right:} Distribution of received SNR across all 40
           designs; the dashed line marks the \SI{10}{\deci\bel}
           requirement.}
  \label{fig:trade}
\end{figure*}

% ======================================================================
\section{Array Layout and Element-Level Analysis}
\label{sec:layout}

The physical arrangement of the $8\times8$ array and the per-element
coupling behavior are summarized in Fig.~\ref{fig:layout}. The left
panel shows the element positions on a $\lambda/2$-spaced rectangular
grid, with marker color indicating the Taylor taper amplitude weight
assigned to each element. The characteristic center-weighted
distribution is clearly visible, with corner elements receiving
approximately 10\% of the weight of center elements.

The right panel reports the active reflection coefficient magnitude for
each of the 64 elements, sorted by their linear index. The mean value
of $\overline{|\Gamma|} = 0.265$ is marked by the dashed line, and the
VSWR $= 2{:}1$ threshold ($|\Gamma| = 0.33$) is shown as a dotted
reference. The majority of elements operate below this threshold,
indicating acceptable impedance match in the array environment. The
single element exhibiting $|\Gamma| = 0.72$ is an interior element that
accumulates constructive coupling contributions from multiple neighbors;
in practice, this would motivate an element-level matching-network
adjustment or a modified coupling model incorporating measured data.

\begin{figure*}[!t]
  \centering
  \includegraphics[width=\textwidth]{fig6_array_layout.png}
  \caption{Element-level analysis of the $8\times8$ array.
           \textbf{Left:} Array layout showing Taylor taper amplitude
           weights by color.
           \textbf{Right:} Active reflection coefficient per element
           index, with mean $|\Gamma|$ (dashed) and VSWR $2{:}1$
           threshold (dotted).}
  \label{fig:layout}
\end{figure*}

% ======================================================================
\section{Conclusion}
\label{sec:conclusion}

This paper has demonstrated a complete, integrated analysis pipeline for
a 28\,GHz phased-array antenna, spanning unit-cell impedance modeling
through system-level link-budget evaluation and multi-objective trade
study. The key quantitative findings for the baseline $8\times8$
configuration are as follows.

The full-wave FEM simulation (EdgeFEM) with Floquet periodic boundary
conditions characterizes the unit-cell surface response on a
\SI{0.254}{\milli\meter} RO4003C substrate, with the Floquet reflection
phase confirming the resonance at \SI{28.5}{\giga\hertz}---within 0.4\%
of the \SI{28.4}{\giga\hertz} predicted by the analytical cavity model.
The feed-port analysis yields a $-10$\,dB impedance bandwidth of
\SI{800}{\mega\hertz}, providing adequate frequency coverage for a
\SI{400}{\mega\hertz} 5G~NR channel. The Taylor-tapered array produces
\SI{10.48}{dBi} of directivity at broadside with negligible scan loss at
\ang{15}. Mutual coupling, modeled via a 64-port distance-dependent
scattering matrix with a mean off-diagonal level of
$-\SI{20.3}{\deci\bel}$, results in a mean active reflection
coefficient of 0.265 and a mismatch loss of only \SI{0.40}{\deci\bel},
with no scan-blindness conditions detected. The complete link budget at
\SI{200}{\meter} range yields an EIRP of \SI{30.1}{dBW}, a received
SNR of \SI{37.6}{\deci\bel}, and a link margin of \SI{27.6}{\deci\bel}
above the \SI{10}{\deci\bel} threshold. The 40-point LHS trade study
reveals that all explored configurations in the
$N_x \in [4,16]$, $N_y \in [4,16]$,
$P_\mathrm{elem} \in [\SIrange{10}{500}{\milli\watt}]$ space
exceed the required SNR, with EIRP ranging from \SI{22.3}{dBW} to
\SI{43.7}{dBW} and costs spanning \$1,600 to \$25,600.

The entire analysis was executed within the APAB open-source toolkit,
which orchestrates the individual analysis steps as MCP tool calls
suitable for LLM-driven automation. Future work will replace the
distance-dependent coupling model with measured or full-wave-simulated
S-parameters, extend the trade study to include substrate thickness,
taper type, and scan-angle requirements as additional design variables,
and incorporate measured antenna patterns for element-pattern
correction in the array-factor computation.

% ======================================================================
% References
% ======================================================================
\begin{thebibliography}{10}

\bibitem{balanis2016}
C.~A. Balanis, \emph{Antenna Theory: Analysis and Design}, 4th~ed.
Hoboken, NJ, USA: Wiley, 2016.

\bibitem{mailloux2017}
R.~J. Mailloux, \emph{Phased Array Antenna Handbook}, 3rd~ed.
Norwood, MA, USA: Artech House, 2017.

\bibitem{apab2026}
J.~A.~Hodge, ``APAB --- Agentic Phased Array Builder,'' 2026. [Online].
Available: \url{https://github.com/jman4162/agentic-phased-array-builder}

\bibitem{hammerstad1975}
E.~O. Hammerstad, ``Equations for microstrip circuit design,'' in
\emph{Proc. 5th Eur. Microw. Conf.}, Hamburg, Germany, Sep. 1975,
pp.~268--272.

\bibitem{bahl1980}
I.~J. Bahl and P.~Bhartia, \emph{Microstrip Antennas}.
Dedham, MA, USA: Artech House, 1980.

\bibitem{taylor1955}
T.~T. Taylor, ``Design of line-source antennas for narrow beamwidth
and low side lobes,'' \emph{IRE Trans. Antennas Propag.}, vol.~AP-3,
no.~1, pp.~16--28, Jan. 1955.

\bibitem{pozar1994}
D.~M. Pozar, ``The active element pattern,'' \emph{IEEE Trans.
Antennas Propag.}, vol.~42, no.~8, pp.~1176--1178, Aug. 1994.

\bibitem{pozar1984}
D.~M. Pozar and D.~H. Schaubert, ``Scan blindness in infinite phased
arrays of printed dipoles,'' \emph{IEEE Trans. Antennas Propag.},
vol.~32, no.~6, pp.~602--610, Jun. 1984.

\bibitem{mckay1979}
M.~D. McKay, R.~J. Beckman, and W.~J. Conover, ``A comparison of
three methods for selecting values of input variables in the analysis
of output from a computer code,'' \emph{Technometrics}, vol.~21,
no.~2, pp.~239--245, 1979.

\bibitem{edgefem2026}
J.~A.~Hodge, ``EdgeFEM --- 3-D finite-element electromagnetic solver,''
2026. [Online]. Available: \url{https://github.com/jman4162/EdgeFEM}

\end{thebibliography}

\end{document}
